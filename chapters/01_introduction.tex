% !TeX root = ../main.tex
% Add the above to each chapter to make compiling the PDF easier in some editors.

\chapter{Introduction}\label{chap:introduction}

**** FIX BIBLIOGRAPHY, not loading for some reason *****

Trends in the automotive industry are shaping the future of cars in a way that electronic and computing devices are becoming increasingly important. In fact, the majority of innovations in the automotive sector are related to electronic systems, either in the form of hardware or software~\parencite{ey1}. Developments such as vehicle electrification, autonomous driving and vehicle connectivity are only some examples of automotive applications where computer processing has a big relevance~\parencite{pwc1}. It is therefore reasonable to seek for approaches to ensure their usage is efficient and safety requirements are met.

In modern cars, dozens of computers, commonly known as electronic control units (ECUs), execute the various computing tasks present in a car. This number is likely to increase if we consider past trends: as the tasks performed by ECUs often have safety-critical constraints, such as real-time capabilities, they are commonly integrated to serve a specific purpose, avoiding conflicts caused by the parallel execution of other tasks~\parencite{vipin1, vipin2}. However, this strategy has shortcomings in the form of inefficient usage of the electronic devices (many tasks are only executed in specific and rare situations), reduced fault tolerance if a system fails, and increased weight and cost of the vehicle due to the high number of ECUs and cables~\parencite{vipin2, baunach1}. Hence, it is an important topic in the automotive industry to find solutions for these issues, especially optimizing costs, while ensuring vehicle safety is kept~\parencite{mckinsey1}.

ECU consolidation is an approach for the reduction in the number of electronic devices in the car. The idea is to consolidate the execution of the tasks from many single-purpose ECUs to a few powerful, multi-purpose ECUs. However, the implementation of ECU consolidation raises other challenges: as more tasks need to be executed on the same platform, higher computation and safety requirements must be met. Regarding safety, it is important to consider the added complexity, and cases where a hardware or software failure occur need to be considered, since a single failure could block many tasks and cause major issues. Also, some safety-critical tasks require redundancy to ensure their correct execution~\parencite{mundhenk1}. These challenges have motivated previous research at the chair of operating systems of the Technical University of Munich, where research on projects such as KIA4SM and MaLSAMi has explored the concept of dynamic task migration as a process where the execution of any tasks could move at any given time from one ECU to another. This would allow tasks to finish execution in case an ECU is overloaded or stops working, effectively allowing them to finish executing and meet their real-time constraints.

Previous work at the chair has divided the task migration process in two stages: planning and execution. The planning is the stage which generates a task distribution to the devices that will execute them. The execution is then responsible for allowing tasks to migrate from one device to another, while ensuring their progress is not lost and the execution continues and finishes correctly at the target device. So far, the work performed at the chair has explored the approach with a single criticality level, using different priority strategies for scheduling the tasks on an ECU (that is, deciding which task should get the processor and be executed at a given time). While this idea is in itself an important contribution, it is relevant to note that in many safety-critical embedded applications, such as the aerospace and automotive industries, different criticality levels exist, which are often defined in standards such as the ISO 26262, which defines the ASIL (automotive safety integrity level). For example, in a car, the correct functioning of the ABS is highly critical, as a failure could be fatal. In contrast, the functionality of the radio system has a lower criticality, as an eventual failure would not cause any issue other than an unpleasant trip. This idea is also important because in these industries there exist certification agencies that validate a system before it can be distributed.

For this reason, and due to the relevance of the concept of mixed-criticality in many industries, it would be important to expand the research on task migration to explore its integration. In particular, expanding the execution stage to ensure tasks with higher criticality are always able to execute properly and meet their deadline is crucial. But also it is important to find a balance between resource efficiency and reliability of the migration process for different criticality levels. Therefore, the work proposed as part of this PhD program aims to implement a strategy for the migration execution at different levels of criticality. Additionally, the mixed-criticality concept has to be integrated as well into the planning stage. This will be also explored in the scope of this research, although the main focus will be in the execution stage.

Furthermore, another important trend is the integration of edge technologies (known as multiaccess edge cloud or mobile edge computing - MEC - or in the specific automotive use-case vehicular edge cloud - VEC- ) for augmenting the vehicle's capabilities in aspects such as communication with other traffic participants or information services, increased sensing capabilities, offloading of tasks, etc (add reference). In the scope of this dissertation, it is considered that edge computing resources can be used to provide a further execution possibility for non-critical tasks under 2 circumstances: 1) the task would benefit from the availability of more powerful computing resources and sensing inputs, and 2) the physical hardware is at a critically high task load level and low criticality tasks would otherwise starve. It is considered that for the most highly critical tasks (like the ones interacting directly with the control of actuators like steering or vehicle speed) this approach is not feasible, as communication with an external component is always prone to interruption, even with the low latencies provided by new communication technologies such as 6G or 5G (add ref).

Although the work performed as part of this research is based on previous developments at the chair and will likely extend existent tools, it includes the development of a platform that implements mixed-criticality scheduling, as well as a migration system that involves the concept in different areas in the planning and execution stages. This includes the selection of hardware and software, as well as network interfaces. Also, verification of the algorithms used will be performed to ensure safety and timing requirements are met.

\section*{Research Questions}
To better define the scope and goal of the research performed in this thesis, following research questions have been propsed as basis (*** Maybe remove and integrate better into Intro text****):

\begin{itemize}
	\item 1. To what extent is task migration feasible in such a distributed real-time system while respecting real-time constraints, and under what conditions?
	\item 2. Given the system's need to handle uncertainty and adapt to changes in hardware availability and task load, how should tasks be prioritized based on criticality? 
	\item 3. What migration approaches - both at communication level and at OS level - are most suitable for high- and low-criticality tasks considering a trade-off between minimizing downtime and optimizing resources?
	\item 4. How feasible is extending the system by offloading tasks to virtual edge resources? What are the implications of this approach, especially in terms of latency, response time, safety and resource optimization?
\end{itemize}

---- See how to integrate into text but:
Following chapters / aspects help find an answer to the research questions:
Q 1: ...
Q 2: Approach, Base OS
Q 3: Migration,  
Q 4: Base OS, Results

\section*{Related Work}\label{section:relatedwork}
At the chair, previous research projects, such as KIA4SM and MaLSAMi, have explored the possibility of migrating tasks running on a device to another in a real-time capable system (for example, ECUs in a vehicle) under certain conditions. As mentioned before, in these works, the migration was divided into two main stages: The first is the migration planning, which determines the hardware that tasks will be migrated to, should the original hardware not be able to fulfill its duty (for example, if there is a failure in that hardware or if the real-time constraints or deadlines would be violated). The second is the execution of the migration, which ensures that corresponding tasks can be migrated from the source device to the target device while keeping their current state.

The migration execution has been explored previously at the chair, for example in projects KIA4SM and HaCRoM. This was explored in the form of a real-time checkpoint-restore mechanism. This involves creating and storing a snapshot of running tasks in a shared memory or copying the memory from a device to another. The work is based on Fiasco.OC and Genode OS. 

MaLSAMi and subsequent theses analyzed migration planning, with researchers performing schedulability analysis based on machine learning (specifically, on neural networks). Machine learning was picked over traditional mathematical approaches such as the ones proposed by Buttazzo~\parencite{buttazzo1}, because the recurrent calculations can become too complex for complex tasks and for big task sets, and they often lead to a pessimistic calculation of the system utilization. By predicting the feasibility of a task set using machine learning algorithms, potentially faster but less precise results are obtained, as demonstrated by previous theses by Taieb~\parencite{taieb1}, Utz~\parencite{utz1} and Blieninger~\parencite{blieninger1}. The predictions provided by the machine-learning approach indicate whether a task set is 100\% schedulable or not, but they are not completely safe, since false positive predictions may occur. This approach could be a potentially powerful solution for enabling the execution at run-time of the real-time capable migration planning. 

Additionally, a few different platforms and setups have been explored in these projects. The used operating systems running on the ECUs are Genode OS, as used in MaLSAMi and a few theses, a real-time operating system based on an extension of Genode OS with Fiasco.OC, as used in KIA4SM~\parencite{kia1} and HaCRoM, and FreeRTOS, as used by Delgadillo~\parencite{delgadillo1}. These developments are considered in the selection of the platform.

These projects have achieved research-relevant results in their segments, but the concept of mixed-criticality has not been explored in related chair internal work. It is therefore necessary to look at relevant chair-external work. In particular, those regarding mixed-criticality scheduling strategies and related to task migration are reviewed next.

It is worth noting that to my knowledge, research on mixed-criticality task migration of high criticality tasks, especially applied to the idea of ECU consolidation, has not been published. A possible reason for the lack of research in this area is the fact that nowadays safety is valued much higher than resource efficiency, thus overseeing the potential for optimization in this aspect. However, in my opinion, it should be possible to achieve both goals by implementing a holistic strategy and therefore it is an area where research can contribute importantly.

