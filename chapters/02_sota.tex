% !TeX root = ../main.tex
% Add the above to each chapter to make compiling the PDF easier in some editors.

\chapter{State of the Art}\label{chap:sota}
*** NEED TO ORGANIZE BETTER MY LITERATURE AND RELATED WORK ***

\section{Automotive Embedded Devices / Multicore}
\begin{itemize}
	\item Add reference of current automotive architectures / AUTOSAR. Explain relevance of multicore approaches - AMP vs SMP
	\item Add references related to traditional hard real-time scheduling -- not considering Mixed Criticality. (Davis \& Burns, Buttazzo)
\end{itemize}

\section{Real-Time Distributed Systems}
\begin{itemize}
	\item
\end{itemize}

\section{Mixed Criticality Scheduling}
Until last decade, the concept of mixed criticality and its involvement in the scheduling of tasks has been mostly explored for single processor systems. In 2011, Baruah et al. proposed an adaptive mixed-criticality scheduling algorithm which set different WCET for different criticality levels of the same task, changing to a higher criticality mode if a job would not report completion after its low criticality deadline~\parencite{baruah2}. Then, also in 2011, Baruah et al. proposed a mixed-criticality scheduling algorithm called EDF-VD for a single processor, which introduces a high criticality mode that modifies high criticality task deadlines to give them a higher priority over low criticality task~\parencite{baruah1}. This algorithm is defined for any number of criticality levels. In 2012 Li and Baruah proposed an extension of the EDF-VD scheduling algorithm to multiprocessors by adding fpEDF strategy, which adds higher priority to tasks with utilization values higher than 0.5~\parencite{libaruah1}. Their approach did not explore migration strategies, but was one of the first implementations of a mixed-criticality scheduler on a multiple processor system. In 2015, Gratia et al. adapted RUN, a global multiprocessor scheduling algorithm, to support mixed criticality~\parencite{gratia1}. This approach also considers only a dual criticality system, and only considers the migration of low criticality tasks.


\section{Task Migration (considering Mixed Criticality)}
In 2018, Ramanathan and Easwaran investigated an approach for mixed-criticality scheduling on multiple processors that involved partial migration based on fixed partitions for some tasks and leaving some low criticality tasks free to execute on any available resource~\parencite{ramanathan1}. In 2019, Zeng et al. tried a similar approach~\parencite{zeng1}. However, their approaches only considered dual-criticality; alsoonly low criticality tasks would migrate from one processor to another, to ensure these tasks also get execution time, even when processors enter high criticality mode and focus on finishing high criticality jobs. In the case of high criticality tasks, they were mapped statically to a single processor.
¨
\section{Vehicular Edge Cloud}
\begin{itemize}
	\item Add reference of survey study by Raza et al.
	\item Add reference to Mark8s as base ofr the containerized approach
	\item Add reference of cloud related task migration (some in Ruano, some in paper list)
	\item Add reference of current MEC technology / 5G / 6G and communication latencies. This will be important to justify assumptions in the process for migration to virtual devices
\end{itemize}



*** KEEP FOR REFERENCE commands...***
Example citations~\cite{barham2003xen, LIS}.
Example acronym usage \gls{cpu}.


	

